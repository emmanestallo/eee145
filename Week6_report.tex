\documentclass[conference]{IEEEtran}
\IEEEoverridecommandlockouts

\usepackage{cite}
\usepackage{amsmath,amssymb,amsfonts}
\usepackage{algorithmic}
\usepackage{graphicx}
\usepackage{textcomp}
\usepackage{xcolor}
\usepackage[american]{circuitikz}
\usepackage{float}

\def\BibTeX{{\rm B\kern-.05em{\sc i\kern-.025em b}\kern-.08em
    T\kern-.1667em\lower.7ex\hbox{E}\kern-.125emX}}
\begin{document}

\title{EEE145 Simulation: Small Motors}

\author{\IEEEauthorblockN{Emmanuel Jesus R. Estallo}
\IEEEauthorblockA{\textit{University of the Philippines Diliman} \\
\textit{Electrical and Electronics Engineering Institute}\\
Quezon City, Philippines\\
emmanuel.estallo@eee.upd.edu.ph}}

\maketitle


\section{Introduction}
\noindent 
Small motors are used for low power applications. Depending on certain conditions, a small motor can 
be operated by both AC and DC sources. The main problem when dealing with AC sources is that we want the
current and flux to change simultaneously to produce torque. In this report, we will compare two 
single-phase motor implementations: capacitor-start and capacitor-start, capacitor-run, using simulink. 

\vspace{8pt}
\section{Simulation: How heavy can you lift?}
\noindent 
The rated torque input, $T_{m}$ is obtained by
\begin{equation*}
T_m = \frac{P_{out}}{\omega}
\end{equation*}
\noindent 
From the given parameters, $P_{out} = 186.5W$. Since the motor has $4$ poles, 
\begin{align*}
\omega_S &= \frac{120f}{p} = 1800 \\
\omega &= \omega_S (1-0.05) = 1710 \cdot \frac{2\pi}{60} = 179.0708 \; rad/s \\ 
T_m &= \frac{186.5}{179.0708} \\ 
&= 1.0415
\end{align*}
\noindent 
The rated torque input, $T_m$ is equal to \framebox{$1.0415$ N-m}.








\end{document}
